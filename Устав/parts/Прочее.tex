\section{Прочее}

\subsection{Самоцветы}

Полезные вложения в порядке убывания полезности:
\begin{enumerate}
    \item ячейка работы в школе за 390 кристаллов (позволяет работать школе вообще без перерыва);    
    \item раскошная ферма (в Греции стоит 500 кристаллов и грейдится за них же) -- достаточно всего одной (нужна для ускорения прохождения ивента);
    \item ячейка рабочего в порту за 190 кристаллов;
    \item дальше по мере накопления расширения территории.
\end{enumerate}
Ещё до Минойской эры включительно можно тратиться на прохождение вглубь Сокровищ на 3 уровне. Но это скорее для помощи Альянсу, а не себе. 

Покупать клетки за кристаллы выгодно только 2 раза (первая 190, вторая 290, а потом уже 490).

\subsection{Товары}

Постоянно производите свой основной товар.
Используйте для этого все возможные мастерские.
И старайтесь торговать только внутри Альянса (это невозможно при переходе между эпохами, описанном в \ref{subsection:treasure_between_epochs}).
Так у нас будет формироваться хорошая скидка друг для друга.
А это по сути дополнительное производство товара.

Никогда не производите товары прошлый эпох.
Пи любой возможности скупайте товары следующей для себя эпохи у товарищей по Альянсу.
Это очень помогает при переходе к следующей эпохе.

\subsection{Двойной сбор}

Не забывайте про механику двойного сбора.

Когда вы получаете жетоны улучшения или роста эпохи особого здания, не применяйте их сразу.
Применять нужно после ежедневного сбора ресурсов с них.
После прокачки Вы сразу же получите следующий сбор.

То же работает и с Чудесами. 
То есть отправлять строителей на улучшение стоит после сбора, если Чудо вообще даёт какой-то сбор, конечно.
