\section{Бой}

\subsection{Расстановка}

В любом бою старайтесь расставить войска максимально плотно, чтобы минимизировать число пустых клеток.

На переднем крае всегда должна быть пехота (обычная или тяжёлая).

Лучники вегда должны быть сзади.
Если у врага есть конница, лучников необходимо ставить либо за двумя слоями пехоты (в основном это касается компании -- в Сокровищах таких больших карт нет), либо за тяжёлой пехотой (но с соблюдением правила максимизации плотности расстановки).

Если у врага дальнобойные войска закрыты тяжёлой пехотой, конников нужно ставить сзади.
Особенно это касается боёв с двумя волнами.
Так вы сможете сэкономить коней на вторую волну.

И если на второй волне остались только лучники, ставьте их в дальнем углу.
За время, пока войска бегут друг к другу, у Вас почти успеет активироваться Командир (щит для лучников).

\subsection{Командиры}

Всех командиров, призывающих войска нужно сбрасывать непосредственно на головы врага.
Так они наносят дополнительный урон, но только если готовы принять выбранную Вами цель в качестве своей.

Не всегда стоит торопиться бросать командиров с войсками в бой.
Иногда бывает так, что у вас остаются лучники и какая-то горстка войск.
Лучше подождать пока враг уничтожит эти войска, и сразу следом кидать новые.
Так у лучников появляется больше времени на нанесение урона.

\textbf{Минос}

Теоретически Минос, будучи конницей, должен оставаться на заднем фланге врага и бить лучников с артиллерией.
Но это не совсем так.
От лучников он чаще всего бежит обратно к пехоте.
А вот артиллерию всегда принимает, как цель.
Поэтому, если она есть, Миноса нужно сбрасывать именно на неё.
А если нет, то на пехоту.
Исключение составляет только случай наличия у врага фортификации.
Если бросить Миноса за ограждение, ему уже некуда будет бежать, и он будет биться с теми, кто за ней стоит.

\textbf{Леонид}

А вот Леонид ведёт себя правильно и понятно.
Он дерётся там, куда его поставили.
Поэтому кидать его нужно всегда на головы лучников.
Именно лучников, а не артиллерии.
Потому что артиллерию быстрее уничтожает конница (Миноса придётся подождать: он регенерируется чуть дольше Леонида),
а лучников эффективнее бьёт пехота.
