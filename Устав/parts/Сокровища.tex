\section{Охота за Сокровищами} 

\subsection{Прохождение боями во время перехода между эпохами}
\label{subsection:treasure_between_epochs}

Делать так можно не часто, но призы в конце стоят того. 
Перед переходом в следующую эпоху затормозите на некоторое время (подробнее в \ref{section:epochs}). 
Копите ресурсы, торгуйте, проходите сокровища в обычном темпе. 
Колбы сливайте в Чудеса членов нашего Альянса.
И не открывайте сундуки! Их вы откроете (постепенно), когда перейдёте в новую эпоху. 
Главное -- набрать достаточное количество ресурсов, чтобы в новой эпохе разом (за день) прокачать хотя бы 2 рода войск. 
Также имеет смысл проходить компанию до упора: новые командиры -- это ваше решающее преимущество.

Когда вы накопите достаточно ресурсов, ждите начала новых сокровищ. 
И сразу, как начнёте их проходить, переходите в новую эпоху.
Дело в том, что сложность боёв выставляется в начале Сокровищ.
Так вы сможете без единого самоцвета и переговоров легко пройти все 60 боёв.

\subsection{Прохождение боями с накоплением улучшений}

Первые два уровня легко проходятся без всяких усилений.
Возможно последние 5 битв придётся проходить переговорами.
Но они довольно лёгкие.

А теперь представим себе ситуацию: Вы копите улучшения и не ставите их
довольно продолжительное время. Больше всего можно поставить домов с
черепушкой.

Дома черепушки попадаются в двух сундуках-чек-поинтах, которые мы всегда берём (с вероятностями 10\% и 20\%),
а также на боях с номерами 30, 39, 40, 44, 54 и 60 (с вероятностью 20\% на каждом).
Бои 54 и 60 можно сразу отбросить. До них сложно дойти. Если есть возможность -- отлично!
Но, если доходить до 44, в среднем за неделю выпадает 1.1 дом с черепушкой.
За это же время выпадает 0.9 улучшений конницы,
0.5 улучшений тяжёлой пехоты,
0.6 улучшений артиллерии (которую в Греции некуда девать),
0.8 улучшений луков.
Но важны по сути только дома с черепушкой. Их в конце Греции 18 (и каждый даёт по 5\% усиления пехоте).
Значить собирать их всех придётся примерно 4.1 месяца.

Это конечно даст 90\% усиления пехоте в Греции, но этого и не нужно.
От Минойской эры до Греции пехота усиливается приблизительно на 15\%. 
От Греции до Древнего Рима -- примерно на 20\%. Это значит, что 6 домов с черепами (30\% усиления пехоты) должно хватать для простого прохождения Сокровищ только боями. 
А это примерно 5.5 недель (предполагается, что Зевс даёт хотя бы 10\%).
Уже более реальная цифра, чем 4 месяца.

Но для более плавного и спокойного прохождения лучше подождать 8 домов-черепушек.
Дело в том, что максимальное улучшение на пиратах — 40\%.
При переходе к новой эпохе у войск растёт не только атака, но и жизнеспособность.
Поэтому нужно компенсировать жизнеспособность атакой.
Однако разницу в 10\% должно быть возможно возместить наличием у нас Командиров-способностей.

В конце Греции на малых и средних объектах культуры можно разместить:
\begin{itemize}
    \item 20\% на коней или тяжёлую пехоту,
    \item 35\% на артиллерию или луки.
\end{itemize}
Строго необходимо разместить все 20\% на коней и 35\% на луки.
При этом ничего не мешает ставить в любое время улучшения тяжёлой пехоты и артиллерии.
Кроме того во время ивента так же можно экономить улучшения войск.

В итоге.
Рецепт должен быть такими.
Примерно 5 недель (пока не накопится 6 домов с черепашками) не ставить ни одного улучшения армии. Потом поставить всё в начале 3 уровня Сокровищ. Должно хватить на 2 полных прохождения Сокровищ.


\subsection{Переговоры}

Начальные переговоры на 3 товара удобнее всего проходить следующим образом. 
На первом этапе всем предлагаем монеты. 
На втором -- еду. 
На третьем -- товар.
Проиграть нельзя.

Далее о более сложных переговорах.

На первом этапе нужно выставить все разные товары. 
Если товаров меньше, чем пиратов, можно выставить несколько единиц самого дешёвого товара (обычно это монеты: имеется ввиду субъективная ценность для Вас в текущей ситуации (то чего мало -- дорого)). 
Если товаров больше, чем пиратов, начинать нужно с выставления самых дешёвых товаров.

Цель первых двух этапов определить, что не нужно ни кому.

На втором этапе продолжайте выставлять в первую очередь те товары, которые ещё не выставляли. 
Во вторую те, которые ещё ни кто не принял. 
Далее опять-таки самый дешёвый товар.

На третьем этапе некоторые пираты уже приняли товар, а для некоторых становится возможно однозначно определить их выбор по тому, что они отвергли. 
Их выставляйте в первую очередь и приравнивание к уже принявшим. 
Затем нужно посчитать какие товары из оставшихся пока не были приняты, какие были приняты единожды, дважды... 
Таким образом получается частота принятия $\nu_i \geqslant 0$ товара с номером $i$.
Очевидно, что чем больше на данный момент частота принятия некоторого товара, тем меньше вероятность того, что он ещё кому-то нужен. 
Так и выставляйте: от наименьшей частоты  к наибольшей.
Если $\nu_{i_1} \leqslant \nu_{i_2} \leqslant \ldots \leqslant \nu_{i_k}$, то выставлять товары в очерёдности $i_1, i_2, \ldots, i_k$. 

То есть, на пример, сначала находим пирата, которому ещё можно предложить товар, который ещё ни кем не принят (частота принятия $\nu_{i_1} = 0$). Предлагаем и засчитываем товару частоту принятия $\nu_{i_1} = 1$. Для следующего пирата (предположим) уже нет товаров с частотой принятия 0. Тогда выбираем для него товар с частотой $\nu_{i_2} = 1$ и присваиваем данному товару частоту $\nu_{i_2} = 2$\ldots

Обязательно следите за тем, что говорят вам пираты о товарах, которые нужны другим пиратам.
Не обращайте внимания на то, что они отвечают последовательно.
Если на третьем этапе у вас осталось 5 возможных товаров, но вы угадали, что нужно всем кроме двух пиратов, а эти пираты говорят, что товар нужен другому пирату, значит, не смотря на то, что, например, этот товар именно сейчас подошёл пятому пирату, который ответил последним, имеется ввиду, что он нужен именно тому, кого вы не угадали.
Просто поменяйте товары местами и вы выиграли.
Так же можно размышлять и для 3 товаров, но уже с некоторой вероятностью неуспеха.

Теперь самое интересное -- переговоры с колбами. 
Не начинайте переговоры без колб, но не спешите их отдавать. 
Первые два этапа переговоров считайте, что колбы ни кому не нужны. 
И только на третьем этапе в случае, если становится однозначно известно, что какому-то пирату нужна именно колба, давайте её ему.

\subsection{Сундуки-чек-поинты}

Самые вкусные награды лежат в сундуках-чек-поинтах. Все 12 мы брать не сможем. 
Для этого все 20 членов должны пройти по 60 битв. Это фантастика. 
Но мы можем более или менее стабильно брать по 9 сундуков. 
Для этого нам нужно собираться пачками по 5-6 человек, которые готовы пройти все 60 битв по одной из представленных выше стратегий.

Уже в 9-ом сундуке с 35\%-ной вероятностью лежат 2 жетона Пиратской Крепости.

Можно искуственно увеличивать себе количество попыток в Охоте за Сокровищами один раз в две недели.
Для этого одну неделю нужно вообще не брать сундуков-чек-поинтов.
И не брать их даже после конца текущей Охоты.
Открывать их нужно после 16:00.
В это время, положенные Вам стартовые 4 попытки сами регенерируются.
После открытия сундуков можно получить ещё 20 попыток и начать следующую охоту с 24-мя попытками!
Это позволит легко пройти первый уровень в первый день.

Лучше всего синхронизировать недели с товарищами по Альянсу.
Так мы сможем очень быстро открывать третий уровень.

Сразу скажу, что во время тестирования данной стратегии я прошёл 59 битв, не открывая сундуков.
