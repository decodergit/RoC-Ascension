\section{Ивенты}

Довольно важный вопрос: какие сундуки (амфоры, \ldots) открывать?
Рассмотрим ситуацию на примере ивента <<Кельты 2022 г.>>.
Все большие ивенты в игре имеют одинаковую механику, так что в дальнейшем стратегия их прохождения не поменяется.

Всего примерно 76500 очков. 
Возьмём среднюю стоимость сундуком по 265, 590 и 765.
В любых ивентах это будет соответствовать маленькому, среднему и большому сундукам.

Если считать по вложениям в основное здание ивента, то получается так:
\begin{itemize}
    \item По 265  $\Rightarrow$ 288 открытий по 1 $\Rightarrow$ 9,6 жетонов;
    \item По 590 $\Rightarrow$ 129 открытий по 2  $\Rightarrow$ 8,6 жетонов;
    \item По 765 $\Rightarrow$ 100 открытий по 3 $\Rightarrow$ 10 жетонов.
\end{itemize}
Выгоднее открывать большой сундук.
Средний сундук открывать вообще не выгодно.

Если считать количество выпадений ивент-зданий, то получается так:
\begin{itemize}
    \item По 265 $\Rightarrow$ 288 открытий по 5\% $\Rightarrow$ 14,4;
    \item По 590 $\Rightarrow$ 129 открытий по 12\% $\Rightarrow$ 15,48;
    \item По 765 $\Rightarrow$ 100 открытий по 15\% $\Rightarrow$ 15.
\end{itemize}
Незначительно выгоднее открывать средний сундук.

В итоге самый выгодный сундук -- большой. Но не без нюансов.

Всего в ивенте 14 разных зданий и 21 день на выполнение.
За полный сбор всех зданий дают ещё 2 жетона основного здания.
Вероятность получить так все здания выставлена разработчиками впритык.
Не сильно рассчитывайте на то, что это получится сделать.

Проходить имеет смысл следующим образом.
В первый день выложиться по полной, чтобы накопить побольше очков.
Нужно внимательно смотреть, что выпадает в качестве приза дня. 
Если это одно из 14 улучшений, которое Вам ещё не выпадало, то нужно открывать сундуки, пока оно не выпадет. 
Иначе -- копить. 
Остаток очков тратить в последний день.

НО! Жетоны улучшения эпохи особых зданий нужно брать ОБЯЗАТЕЛЬНО в ЛЮБЫХ доступных количествах! 
